\chapter{Meshes}
\label{sec:mesh}
\ippl predefines classes to represent Cartesian meshes; these typically serve as the \texttt{Mesh} template parameter for \texttt{Field} 
and other classes parameterized on \texttt{Mesh} type. These classes also provide various mechanism to query mesh 
geometry (spacings, cell volumes, etc.) from \texttt{Mesh} objects. 

\section{Mesh Class} 
The \texttt{Mesh} class is an abstract base class for classes representing computational meshes. 
Currently, the base class does nothing it does not even provide any virtual functions, because it is difficult 
to conceive of commonality among potential derived meshes as disparate as unstructured and uniform structured meshes 
(for example). It does allow writing functions and classes that have \texttt{Mesh} objects as arguments and members; 
but, so far, that hasn't been used even within the
\ippl internal implementation. \ippl predefines the \texttt{UniformCartesian} and \texttt{Cartesian} classes which inherit from \texttt{Mesh}. 

\subsection{Mesh Definition (Public Interface)} 
\begin{smallcode}
template<unsigned Dim> 
class Mesh 
{};
\end{smallcode}

\section{\texttt{UniformCartesian} Class} 
\label{app:unifcart}
The \texttt{UniformCartesian} class represents the abstraction of a uniform-spacing Cartesian mesh discretizing a rectangular region of space. The \texttt{Mesh} has 
uniform spacing in the sense that the mesh spacings (vertex-vertex distances) along a dimension are the same {\it all along that dimension}. The different dimensions 
may have different (single) values for mesh spacing. The \texttt{Cartesian} class (Appendix \ref{app:cartclass}) generalizes this to meshes whose spacings vary cell-by-cell along each 
dimension. \texttt{UniformCartesian} has mechanisms for returning various kinds of geometrical information from the mesh: cell-cell and vertex-vertex spacings, 
nearest mesh vertex positions to a given point in space, and others. Many of these; such as a function to return the volume of a particular indexed cell in the mesh, 
are somewhat redundant, for the uniform case, but make more sense in the nonuniform case (\texttt{Cartesian}); the interfaces of \texttt{UniformCartesian} and 
\texttt{Cartesian} are meant to be as much alike as possible. 
\texttt{UniformCartesian} is parameterized on dimensionality Dim, and another parameter MFLOAT. The MFLOAT parameter specifies the elemental type to use 
in storing and returning mesh geometrical information such as spacings and position coordinates. Generally, this should be a floating-point type, and it defaults 
to double. Vector values are represented using \texttt{Vektor<MFLOAT,Dim>}. 

\subsection{\texttt{UniformCartesian} Definition (Public Interface)}
\begin{smallcode}
template < unsigned Dim, class MFLOAT=double>
class UniformCartesian : public Mesh<Dim>
{ );
public: 
// Public member data: 
unsigned gridSizes[Dim];            // Sizes (number of vertices) 
typedef Cell DefaultCentering;   // used by Field
Vektor<MFLOAT,Dim>  Dvc[1<<Dim];  // Constants for derivatives 

bool hasSpacingFields;  	          
BareField<Vektor<MFLOAT, Dim>, Dim>* VertSpacings ;
BareField<Vektor<MFLOAT, Dim>, Dim>* CellSpacings ;

// Public member functions: 
// Constructors 
UniformCartesian() {}; // Default constructor 

// Non-default constructors 
UniformCartesian(NDIndex<Dim>& ndi); 
UniformCartesian(Index& I); 
UniformCartesian(Index& I, Index& J); 
UniformCartesian(Index& I, Index& J, Index& K); 
// These also take a MFLOAT* specifying the mesh spacings: 
UniformCartesian(NDIndex<Dim>& ndi, MFLOAT* delX); 
UniformCartesian(Index& I, MFLOAT* delX); 
UniformCartesian(Index& I, Index& J, MFLOAT* delX); 
UniformCartesian(Index& I, Index& J, Index& K, MFLOAT* delX); 
// These further take a Vektor<I:1FLOAT,Dim>& specifying the origin: 
UniformCartesian(NDIndex<Dim>& ndi, MFLOAT*delX, Vektor<MFLOAT,Dim>& orig); 
UniformCartesian(Index& I, MFLOAT* delX, Vektor<MFLOAT,Dim> orig); 
UniformCartesian(Index& I, Index& J, MFLOAT* delX, Vektor<MFLOAT,Dim>& orig) ;
UniformCartesian(Index& I, Index& J, Index& K, MFLOAT* delX, Vektor<MFLOAT,Dim>& orig); 

~UniformCartesian() { }; // Destructor 

// Set functions for member data: 
// Create BareField's of vertex and cell spacingsi allow for specifying 
// layouts via the FieldLayout e_dim_tag and vnodes parameters (these 
// get passed in to construct the FieldLayout used to construct the BareField's).
 
void storeSpacingFields(); // Defaulti will have default layout 

// Special cases for 1-3 dimensions, a la FieldLayout ctors
void storeSpacingFields(e_dim_tag pI, int vnodes=-1);
void storeSpacingFields (e_dim_tag pI, e_dim_tag p2', int vnodes=-l) ; 
void storeSpacingFields (e_dim_tag pI, e_dim_tag p2, e_dim_tag p3,int vnodes=-1); 

// It Next we have one for arbitrary dimension, a la FieldLayout ctor: 
// All the others call this one internally: 
void storeSpacingFields(e_dim_tag *p, int vnodes=-1); 

// Accessorfunctions for member data: 
// Get the origin of mesh vertex positions: 
Vektor<MFLOAT,Dim> get_origin(); 

// Get the spacings of mesh vertex positions along specified direction: 
MFLOAT get_meshSpacing(int d); 

// Get the cell volume: 
MFLOAT get_volume(); 

// Formatted output of UniformCartesian object: 
void print (ostream&);
// Stream formatted output of UniformCartesian object: 
friend ostream& operator<<(ostream&, const UniformCartesian<Dim,MFLOAT>&);

// Other UniformCartesian methods

// Volume of single cell indexed by input NDIndex
MFLOAT getCellVolume(NDIndex<Dim>&);

// Field of volumes of all cells: ' Field<MFLOAT,Dim,UniformCartesian<Dim,MFLOAT>,Cel1>& 
getCellVolumeField (Field<MFLOAT, Dim, UniformCartesian<Dim,MFLOAT>,Cell>&); 

// Volume of range of cells bounded by verticies specified by inputNDIndex: 
MFLOAT getVertRangeVolume(NDIndex<Dim>&); 

// Volume of range of cells spanned by input NDIndex (index o'f cells): 
MFLOAT getCe//Rangevolume(NDIndex<Dim>&); 

// Nearest vertex index to (x,y,z)
NDIndex<Dim>& getNearestVertex(Vektor<MFLOAT,Dim>&); 

// Nearest vertex index with all vertex coordinates below (x,y,z): 
NDIndex<Dim>& getVertexBelow(Vektor<MFLOAT,Dim>&);

// NDIndex for cell incell-ctrd Field containing the point (x,y,z): 
NDIndez<Dim>& getCellContaining(Vektor<MFLOAT,Dim>&);
 
// (x,y,z) coordinates of indexed vertex: 	_ 
Vektor<MFLOAT, Dim> getVertexPosition (NDIndex<Dim>&); 
 
// Field of (x,y,z) coordinates of all vertices: 
Field<Vektor<MFLOAT, Dim>, Dim, UniformCartesian<Dim, MFLOAT>, Vert>& 
getvertexPositionField(Field<Vektor<MFLOAT,Dim>,Dim, UniformCartesian<Dim,MFLOAT>,Vert>& ); 
 
//Vertex-vertex grid spacing of indexed cell:
Vektor<MFLOAT,Dim> getDeltaVertex(NDIndex<Dim>&); 

// Field of vertex-vertex ,grid spacings of all cells: 
Field<Vektor<MFLOAT,Dim>,Dim,UniformCartesian<Dim,MFLOAT>,Cell>& 
getDeltaVertexField (Field<Vektor<MFLOAT, Dim>, Dim, UniformCartesian<Dim,MFLOAT>,Cell>& ); 

// Cell-cell grid spacing of indexed vertex: 
Vektor<MFLOAT,Dim> getDeltaCell(NDIndex<Dim>&); 

// Field of cell-cell grid spacingsof all vertices: 
Field<Vektor<MFLOAT, Dim>, Dim,UniformCar,tesian<Dim, MFLOAT>, Vert>& 
getDeltaCellField(Field<Vektor<MFLOAT,Dim>,Dim, UniformCartesian<Dim,MFLOAT>,Vert>& ); 

// Array of surface normals to cells adjoining indexed cell: 
Vektor<MFLOAT,Dim>* getSurfaceNormals(NDIndex<Dim>&); 

// Array of {pointers to} Fields of surface normals to all cells: 
void getSurfaceNormalFields(Field<Vektor<MFLOAT,Dim>,Dim, UniformCartesian<Dim,MFLOAT>,Cell>** ); 

// Similar functions, but specify the surface normal to a single face, using 
// the following numbering convention: 0 means low face of 1st dim, 1 means 
// high face of 1st dim, 2 means low face of 2nd dim, 3 means high face of  2nd dim, and so on: 	, 

Vektor<MFLOAT,Dim> getSurfaceNormal(NDIndex<Dim>&, unsigned); 

Field<Vektor<MFLOAT,Dim>,Dim,UniformCartesian<Dim,MFLOAT>,Cell>& getSurfaceNormalField(
Field<Vektor<MFLOAT,Dim>,Dim, UniformCartesian<Dim,MFLOAT>,Cell>&, unsigned); 
\end{smallcode}


\subsection{UniformCartesian Constructors}
Aside from the default constructor, which you should not be used if you, don't know what it's there for, there are three categories of \texttt{UniformCartesian} constructors. 
All require an argument or set of \texttt{Dim} arguments specifying the number of mesh nodes along each dimension. The one-argument form takes an \texttt{NDIndex<Dim>\&} 
to specify this: the \texttt{length()} value of each \texttt{Index} and the \texttt{NDIndex} represents the number of mesh nodes (vertices); the multi-argument forms take 
\texttt{Dim Index\&}'s (these are implemented only up to Dim = 3). {\it Warning:} be sure to use zero-based, unit-stride \texttt{NDIndex Index} objects; \texttt{UniformCartesian} 
should eventually work for other cases, but for now the implementation doesn't account for non-zero base or non-unit stride for the index space spanning the mesh nodes/cells. 
The first and simplest category of constructors has only the size arguments; here are examples of the one-argument and multi-argument case: 
\begin{smallcode}
Index I {5} , J {5} , K (5 ); 
NDIndex<3> ndi; 
ndi[0] = I; ndi[1] = J; ndi[2] = K;

UniformCartesian<3> umesh1(ndi); 
UniformCartesian<3> umesh2(I,J,K); 
\end{smallcode}

Both of these \texttt{UniformCartesian} objects will have default mesh spacings of 1.0 in all directions and an origin (location of first vertex) at (0.0, 0.0, 0.0). 
The second category adds specification of the mesh spacings. You create an array of type MFLOAT and pass it to the constructor:
 
\begin{smallcode}
double spacings[3] = {1. 0, 1,.0, 2.0}; //double because default of MFLOAT 
UniformCartesian<3> umesh3{ndi, spacings);
\end{smallcode} 

The \texttt{umesh3} object has mesh spacings $\Delta_x= 1.0,~\Delta_y=1.0, \text{ and }\Delta_z = 2.0$. We used type \texttt{double} for the array of spacings because we used the 
default \texttt{MFLOAT} template parameter value of \texttt{double} when instantiating \texttt{umesh3}. If we had specified something else, \texttt{UniformCartesian<3,float>}, we 
would have had to specify \texttt{float} as the type of the spacings array. The \texttt{umesh3} object has a default origin (location of first vertex) at (0:0, 0.0, 0.0). 

The third category adds specification of the origin (location of the first vertex). You create a \texttt{Vektor<MFLOAT, Dim>} and pass it as the last constructor argument: 

\begin{smallcode}
// Defining spacings and origin, use double because it's default .of MFLOAT: 
double spacings[3] = {1.0,1.0,2.0}; 
Vektor<double,3> origin; 
origin(0) = 5.0; origin(1) = 6.0; origin(2)= 7.0; 
UniformCartesian<3> umesh4(ndi, spacings, origin); 
\end{smallcode}
The umesh4 object has mesh spacings $\Delta_x= 1.0,~\Delta_y=1.0, \text{ and }\Delta_z = 2.0$ and origin at (5.0,6.0,7.0).



\subsection{\texttt{UniformCartesian} Member Functions and Member Data} 
\texttt{UniformCartesian} Member Data for Sizes and Spacings. There are several-public data members representing mesh size and spacing information: 
\begin{smallcode}
unsigned gridSizes[Dim] ;
\end{smallcode}
An array containing the mesh sizes-numbers of vertices along each dimension. 
\begin{smallcode}
typedef Cell DefaultCentering ;
\end{smallcode} 
Used by \texttt{Field} and other classes which are parameterized on \texttt{Mesh} and centering classes and require consistent defaults for both. 
The general user probably need never use this member. 
\begin{smallcode}
Vektor<MFLOAT,Dim> Dvc[1<<Dim] ;
\end{smallcode}
Constants for derivatives in global differential operator functions such as \texttt{Div()}. Again, the general user probably never need know that this is publicly visible.  
\begin{smallcode}
bool hasSpacingFields ;
\end{smallcode}
Flags whether the user has requested that the mesh object internally allocate and compute mesh-spacing \texttt{BareField}'s pointed to by \texttt{VertSpacings} and \\ \texttt{CellSpacings}. 
\begin{smallcode}
BareField<Vektor<MFLOAT,Dim>,Dim>* VertSpacings; 
BareField<Vektor<MFLOAT,Dim>,Dim>* CellSpacings;
\end{smallcode}
If you invoke the \texttt{storeSpacingFields()} function, \texttt{UniformMesh} will allocate two \texttt{BareField}'s and fill them with vertex-vertex and cell-cell mesh spacing values 
(stored as vectors whose components are the spacings along each dimension of the cell or shifted cell). These pointers provide public access to them. 
The general user can always use the functions like \texttt{getDeltaVertexField ()} to put spacing values into his own \texttt{Field} objects, and this may be the best way to do this in general. 
Certain predefined \ippl global functions, such as the \texttt{Div()} differential operators, might rely on having these BareField's internally available from the mesh object associated with 
the \texttt{Field}'s on which they operator. 
Note: These internal \texttt{BareField}'s are redundant for \texttt{UniformCartesian} in a couple of ways. First, the mesh spacing information is the same everywhere in the \texttt{Mesh}, 
and the vertex-vertex spacing is the same as the cell-cell spacing. The operators like \texttt{Div()} don't need \texttt{BareField}'s of spacings for the uniform Cartesian case; and they don't, 
in fact, use them or check if they exist. Both of these redundancies are removed in the nonuniform Cartesian case, however. In this case, storing this information in the internal 
\texttt{BareField}'s in the Cartesian objects is essential for functions like \texttt{Div(}) -they return errors if \texttt{hasSpacingFields} is false. 


\subsection{\texttt{UniformCartesian} Set/Accessor Functions for Member Data} 

\begin{smallcode}
void storeSpacingFields ( );
void storeSpacingFields(e_dim_tag pI, int vnodes=-I) 
void storeSpacingFields(e_dim_tag pl, e_dim_tag p2, int vnodes=-1); 
void storeSpacingFields(e_dim_tag pI, e_dim_tag p2, e_dim_tag p3, int vnodes=-1); 
void storeSpacingFields(e_dim_tag *p, int vrlodes=-1);
\end{smallcode}
The \texttt{UniformCartesian} class will optionally create internal \texttt{BareField}'s of appropriate sizes and fill them with vertex-vertex arid cell-cell spacing values. 
You access these \texttt{BareField}'s via the \texttt{VertSpacings} and \texttt{CellSpacings} pointers described above. 
The \texttt{storeSpacingFields()} functions make this happen. Because they are constructing \texttt{BareField}'s, they provide prototypes based on those for \texttt{FieldLayout} 
so you can control the serial/parallel layout of the \texttt{BareField}'s, you specify which dimensions are serial or parallel using lists or an array of \texttt{e\_dim\_tag} 
values (\texttt{SERIAL} or \texttt{PARALLEL}). If you use the first prototype, with no arguments, you will get the default \texttt{FieldLayout} for the internal \texttt{BareField}'s: 
parallel for all dimensions.

\begin{smallcode}
Vektor<MFLOAT,Dim> get_origin();
\end{smallcode}
Returns the value of the origin of the mesh (position in space of the lowest mesh vertex). 

\begin{smallcode}
MFLOAT get_meshSpacing(int d);
\end{smallcode}
Returns the mesh spacing value for the specified direction. There is only one value for uniform spacing; thus the return type \texttt{MFLOAT}. 
\begin{smallcode}
MFLOAT get_volume() ;
\end{smallcode}
Returns the volume of a cell, which is the same everywhere in a uniform Cartesian mesh. 


\subsection{Other UniformCartesian Methods} 
Most of the public member functions in \texttt{UniformCartesian} are designed to return some typical kinds of geometrical information about the \texttt{Mesh} that a user (programmer or class) might want. 
Where the information or input parameters are more complex than single \texttt{MFLOAT} values, the function return values or arguments are typically \ippl classes  such as \texttt{Field}.
The set of functions evolved from general discussions among application programmers of, what is expected of a mesh, and we will continue to evolve its design iteratively as new applications demand new \texttt{Mesh} 
information. Many of these functions use \texttt{NDIndex} to index mesh vertices and cells. \texttt{NDIndex} has no intrinsic awareness of centering, but can clearly represent index values specifying 
mesh node or cell positions. The user must keep in mind that cell 0 along a dimension is between node 0 and node 1, and that there are one fewer cells than vertices. 
\begin{smallcode}
MFLOAT getCellVolume(NDIndex<Dim>&);
\end{smallcode} 
Volume of single cell indexed by input \texttt{NDIndex}. The argument must describe a single element. That is, the range of every \texttt{Index} in the \texttt{NDIndex} must be 1. The \texttt{getCellVolume()} function 
returns an error otherwise.

\begin{smallcode}
Field<MFLOAT,Dim,UniformCartesian<Dim,MFLOAT>, Cell>& 
getCellVolumeField (Field<MFLOAT, Dim, UniformCartesian<Dim,MFLOAT>,Cell>&);
\end{smallcode}
Field of volumes of all cells. This function basically assigns every element of the cell-centered Field to the return value of \texttt{getCellVolume ( )}. For a \texttt{UniformCartesian} mesh, this is obviously redundant,
but the function is here for interface compatibility with \texttt{Cartesian} (which represents a nonuniform Cartesian mesh, for which this is not redundant). 
If you are only using \texttt{UniformCartesian} mesh objects, you should just use the single cell-volume value retuned by \texttt{getCellVolume}; this will combine with other\texttt{ Field}'s of 
values in your code the same way any scalar value will do. 
\begin{smallcode}
MFLOAT getVertRangeVolume(NDIndex<Dim>&);
\end{smallcode} 
Volume of range of cells bounded by vertices specified by input \texttt{NDIndex}, which in this context will generally have a range greater than one in at least one dimension. The vertices represented by the lowest and 
highest index value set contained in the \texttt{NDIndex} mark the corners of a rectangular solid region; this function returns the volume of that region. 
\begin{smallcode}
MFLOAT getCellRangevolume(NDIndex<Dim>&);
\end{smallcode}
Volume of range of cells spanned by input \texttt{NDIndex} (index of cells). This is like \texttt{getVertRangeVolume( )} , except that the corners of the rectangular solid are 
cell-center positions rather than vertex positions. 

\begin{smallcode}
NDIndex<Dim>& getNearestvertex(Vektbr<MFLOAT,Dim>&);
\end{smallcode}
Nearest vertex index to a point in space (x,y,z). 

\begin{smallcode}
NDIndex<Dim>& getVertexBelow(Vektor<MFLOAT,Dim>&);
\end{smallcode} 
Nearest vertex inpex with all vertex coordinates below a point in space (x,y,z). 

\begin{smallcode}
NDIndex<Dim>& getCellContaining(Vektor<MFLOAT,Dim>&) ;
\end{smallcode}
\texttt{NDIndex} for the mesh cell containing the point (x,y,z). Use this, for example, to index a corresponding element in a cell-centered \texttt{Field}. 

\begin{smallcode}
Vektor<MFLOAT,Dim> getVertexposition(NDIndex<Dim>&) ;
\end{smallcode}
(x,y,z) coordinates of the \texttt{Mesh} vertex indexed by the \texttt{NDIndex}, which just have a range of one in all dimensions (that is, it must index a single point in index space). 

\begin{smallcode}
Field<Vektor<MFLOAT, Dim>, Dim, UniformCartesian<Dim,MFLOAT>,Vert>& 
getVertexPositionField(Field<Vektor<MFLOAT, Dim>, Dim, UniformCartesian<Dim,MFLOAT>,Vert>& );
\end{smallcode}
Fills a vertex-centeredField with the (x,y,z) coordinates of all mesh vertices. 

\begin{smallcode}
Vektor<MFLOAT, Dim> getDeltaVertex (NDIndex<Dim>&); 
\end{smallcode}
Vertex-vertex grid spacing $(\Delta_x, \Delta_y, \Delta_z)$ of the cell indexed by the NDIndex. 

\begin{smallcode}
Field<Vektor<MFLOAT, Dim>, Dim, UniformCartesian<Dim,MFLOAT>,Cell>& 
getDeltaVertexField(Field<Vektor<MFLOAT,Dim>,Dim, UniformCartesiap<Dim,MFLOAT>,Cell>& );
\end{smallcode}
Fills a cell-centered Field with the vertex-vertex grid spacings $(\Delta_x, \Delta_y, \Delta_z)$  of all cells. 

\begin{smallcode}
Vektor<MFLOAT,Dim> getDeltaCell(NDIndex<Dim>&) ;
\end{smallcode}
Cell-cell grid spacing $(\Delta_x, \Delta_y, \Delta_z)$ of indexed cell vertex. That is, this returns the distance between the cell centers on either side of the vertex position indexed by the \texttt{NDIndex} for each dimension. 

\begin{smallcode}
Field<Vektor<MFLOAT, Dim>, Dim,uniforrnCartesian<Dim, MFLOAT>, Vert>& 
getDeltaCellField(Field<Vektor<MFLOAT,Dim>,Dim, UniformCartesian<Dim,MFLOAT>,Vert>& ) ;
\end{smallcode}
Fills a vertex-centered Field with,the cell-cell grid spacings $(\Delta_x, \Delta_y, \Delta_z)$ around all vertices. 

\begin{smallcode}
Vektor<MFLOAT,Dim>* getSurfaceNormals(NDIndex<Dim>&) ;
\end{smallcode}
Array of surface normals to cells adjoining indexed cell. This is trivial for a Cartesian mesh, and is the same for every cell even in the nonuniform spacing cases represented by \texttt{Cartesian}. 
Future implementations in non-cartesian mesh classes would be more complicated. 

\begin{smallcode}
void getSurfaceNormalFields(Field<Vektor<MFLOAT,Dim>,Dim, UniformCartesian<Dim,MFLOAT>,Cell>**);
\end{smallcode}
Fills the Field's pointed to by the array with of surface normals to all cells. Again, this is trivial for a cartesian mesh, and the values are the same everywhere; but future implementations in non-cartesian meshes would be more complicated. 

\begin{smallcode}
Vektor<MFLOAT,Dim> getSurfaceNormal(NDIndex<Dim>&, unsigned) ;

Field<Vektor<MFLOAT,Dim>,Dim, UniformCartesian<Dim,MFLOAT>,Cell>& 
getSurfaceNormalField(Field<Vektor<MFLOAT,Dim>,Dim, 
	UniformCartesian<Dim,MFLOAT>,Cell>&, unsigned) ;
\end{smallcode}
Similar functions to \texttt{getSurfaceNormals()} and \texttt{getSurfaceNormalFields()}, but specify the surface normal to a single face, using the following numbering convention: 
0 means low face of 1st dimension, 1 means high face of 1st dimension, 2 means low face of 2nd dimension, 3 means high face of 2nd dimension; and so on. 

\section{\texttt{Cartesian} Class}
\label{app:cartclass} 
The \texttt{Cartesian} class represents the abstraction of a nonuniform-spacing Cartesian mesh discretizing a rectilinear region of space. The mesh spacings vary cell-by-cell along each dimension. As much as possible, the 
\texttt{Cartesian} interface is identical to the \texttt{UniformCartesian} interface described in Section xxx. It has the same mechanisms for returning various kinds of geometrical information from the \texttt{Mesh}: 
cell-cell and vertex-vertex spacings, nearest mesh vertex positions to a given point in space, and others. 
Like \texttt{UniformCartesian, Cartesian} is parameterized on dimensionality \texttt{Dim}, and another parameter \texttt{MFLOAT} which specifies the elemental type to use in storing and returning mesh geometrical 
information such as spacings and position coordinates. Generally, this should be a floating-point type, and it defaults to \texttt{double}. Vector values are represented using \texttt{Vektor<MFLOAT,Dim>}. 
This chapter only discusses the places where the \texttt{Cartesian} interface differs from the \texttt{UniformCartesian} interface. Refer to Section xxx for all other information about \texttt{Cartesian}; substitute "\texttt{Cartesian}" for "\texttt{UniformCartesian}" in places such as the \texttt{Mesh} parameter for \texttt{Field} arguments to the member functions. To help with this, we show the entire \texttt{Cartesian} public definition in the next section; 
in the subsequent sections discussing the member functions and data, we discuss only the cases that differ from \texttt{UniformCartesian}. 

\subsection{\texttt{Cartesian} Definition (Public Interface)}
 Enumeration used for specifying mesh boundary conditions. Mesh BC are used  for things like figuring out how to return the mesh spacing for a cell 
beyond the edge of the physical mesh, as might arise in stencil operations lion Field's on the mesh.
 
\begin{smallcode}

enum MeshBC_E { Reflective, Periodic, None };
char* MeshBC_E_Names"[3] = {"Reflective", "Periodic ", "None"} ;

template < unsigned Dim, class MFLOAT=double> 
class Cartesian : public Mesh<Dim> 
{ 
public: 
// Public member data: 
unsigned gridSizes[Dim]; // Sizes (number of vertices) 
typedef Cell DefaultCentering; //Default cemtering (used by Field,etc.)
Vektor<MFLOAT,Dim> Dvc[1<<Dim];  // Constants for derivatives. 

bool hasSpacingFields; // Flags allocation of the following:
BareField<Vektor<MFLOAT,Dim>, Dim>* VertSpacings;
BareField<Vektor<MFLOAT, Dim>, Dim>* CellSpacings;

// Public member functions: 
// Constructors 
Cartesian() {} ; // Default constructor 

// Non-default constructors 
Cartesian (NDIndex<Dim>& ndi); 
Cartesian(Index& I);
Cartesian(Index& I; Index& J);
Cartesian(Ihdex& I, Index& J, Index& K);

// These also take a MFLOAT** specifying the mesh spacings: 
Cartesian(NDIndex<Dim>& ndi, MFLOAT** delX); 
Cartesian(Index& I, MFLOAT** delX);
Cartesian(Index& I, Index& J, MFLOAT**deIX); 
Cartesian(Index& I, Index& J, Index& K, MFLOAT** deIX); 

// These further take a Vektor<MFLOAT,Dim>& specifying the origin:
Cartesian(NDIndex<Dim>& ndi, MFLOAT** deIX,Vektor<MFLOAT,Dim>& orig); 
Cartesian(Index& I, MFLOAT** delX, Vektor<MFLOAT,Dim> orig); 
Cartesian (Index& I, Index& J, MFLOAT* * delX, Vektor<MFLOAT ,Dim>& orig) ; 
Cartesian(Index& I, Index& J, Index& K, MFLOAT** delX, Vektor<MFLOAT,Dim>& orig); 
 
// These further take a MeshBC_E array specifying mesh boundary conditions, 
Cartesian(NDIndex<Dim>& ndi, MFLOAT** delX, Vektor<MFLOAT,Dim>& arig, MeshBC_E* mbc);
Cartesian (Index& I, MFLOAT* * deIX, Vektor<MFLOAT, Dim> arig, MeshBC_E* mbc); 
Cartesian(Index& I, Iridex& J, MFLOAT** delX, Vektor<MFLOAT,Dim>& orig, MeshBC_E* mbc); 
Cartesian(Index& I, Index& J, Index& K, MFLOAT** delX, Vektor<MFLOAT,Dim>& arig, MeshBC_E* mbc); 
-Cartesian () { };
\end{smallcode}

Set functions for member data: create\texttt{ BareField}'s of vertex and cell spacings; allow for specifying layouts via the \texttt{FieldLayaut e\_dim\_tag} and vnodes parameters 
(these get passed by constructing  the \texttt{FieldLayaut} and used to, construct the \texttt{BareField}' s) . 

\begin{smallcode}

void stareSpacingFields(); // Default will have default layaut 

// Special cases far 1-3 dimensians, ala FieldLayaut ctars: 
void stareSpacingFields(e_dim_tag p1; int vnodes=-1); 
void stareSpacingFields (e_dim_tag pl, e_dim_tag p2, int vnodes=-1); 
void stareSpacingFields(e_dim_tag p1, e_dim_tag p2, e_dim_tag p3, int vnodes=-1); 

// Next we have ane far arbitrary dimensian,a la FieldLayaut ctor: 
// All the others call this one internally: 
void storeSpacingFields(e_dim_tag *p, int vnodes=-1);
 
// Accessar functians far member data: 
// Get the arigin of mesh vertex pasitians: 
Vektor<MFLOAT,Dim> get_orgin(); 
 
// Get the spacings of mesh vertex positions along specified direction: 
MFLOAT* get_meshSpacing(int d); 

// Get mesh boundary conditions: 
MeshBC_E get_MeshBC(unsigned face);     // One face at a time 
MeshBC_E* get_MeshBC () ; 	                        // All faces at ance

// Formatted output of Cartesian object: 
void print(ostream&); 

// Stream formatted output of Cartesian abject:
friend ostream& operato<<(ostream&, const Cartesian<Dim,MFLOAT>&);
 
  
// Other Cartesian methods
  
// Volume of a single cell indexed by input NDIndex
MFLOAT getCellVolume(NDIndex<Dim>&); 
  
// Field of Volumesof of all cells
Field<MFLOAT,Dim,Cartesian<Dim,MFLOAT>, Cell>& getCellVolumeField(
Field<MFLOAT,Dim,Cartesian<Dim,MFLOAT>,Cell>&);
   
// Volume of range of cells bounded by verticies specified by input NDIndex: 
MFLOAT getVertRangeVolume (NDIndex<Dim>&);
    
// Volume of range of cells spanned by input NDIndex (index of cells) 
MFLOAT getCellRangeVolume(NDIndex<Dim>&);
     
// Nearest vertex index to, (x,y,z)
NDIndex<Dim>& getNearestVertex (Vektor<MFLOAT, Dim>&);

// Nearest vertex index with all vertex coordinates below (x,y, z)
NDIndex<Dim>& getVertexBelow(Vektor<MFLOAT,Dim>&); 

// NDIndex for cell in cell-ctrd Field containing the point (x,y, z) 
NDIndex<Dim>& getCellContaining(Vektor<MFLOAT,Dim>&); 

// (x,y,z) coordinates of indexed vertex
Vektor<MFLOAT,Dim> getVertexPosition(NDIndex<Dim>&); 

// Field of (x,y,z) coordinates of all vertices: 
Field<Vektor<MFLOAT,Dim> ,Dim, Cartesian<Dim,MFLOAT>,Vert>& getVertexPositionField(
Field<Vektor<MFLOAT,Dim>, Dim, Cartesian<Dim,MFLOAT>,Vert>&);

// Vertex-vertex grid spacing of indexed vertex
Vektor<MFLOAT,Dim> getDeltaVertex(NDIndex<Dim>&);
 
// Field of vertex-vertex grid spacings of all vertices
Field<Vektor<MFLOAT,Dim> ,Dim, Cartesian<Dim, MFLOAT> , Cel 1>& getDeltaVertexField(
Field<Vektor<MFLOAT,Dim>, Dim, Cartesian<Dim,MFLOAT>,Cell>& );

// Cell-cell grid spacing of indexed cell: 
Vektor<MFLOAT,Dim> getDeltaCell(NDIndex<Dim>&);

// Field of cell-cell grid spacings of all vertices: 
Field<Vektor<MFLOAT,Dim>, Dim, Cartesian<Dim,MFLOAT>,Vert>& getDeltaCellField(
Field<Vektor<MFLOAT,Dim>,Dim, Cartesian<Dim,MFLOAT>,Vert>& );

// Array of surface normals to cells adjoining indexed cell: 
Vektor<MFLOAT,Dim>* getSurfaceNormals(NDIndex<Dim>&);

// Array of (pointers to) Fields of surface normals to all cells: 
void getSurfaceNormalFields(Field<Vektor<MFLOAT,Dim>,Dim, Cartesian<Dim,MFLOAT>,Cell>** );
\end{smallcode}
 Similar functions, but specify the surface normal to a single face, using the following numbering convention: 0 means low face of 1st dim, 1 means 
 high face of 1st dim, 2. means low face of 2nd dim, 3 means high face of 2nd dim, and so on:
 \begin{smallcode}
Vektor<MFLOAT,Dim> getSurfaceNormal(NDIndex<Dim>&, unsigned);
Field<Vektor<MFLOAT,Dim>,Dim, Cartesian<Dim, MFLOAT> , Cel 1>& getSurfaceNormalField(
Field<Vektor<MFLOAT,Dim>,Dim, Cartesian<Dim,MFLOAT>,Cell>&, unsigned);
} ;
\end{smallcode}



\subsection{Cartesian Constructors} 
Aside from the default constructor, which you should not be using if you don't know what it's there for, there are four categories of \texttt{Cartesian} constructors. 
The first three are the same as for \texttt{UniformCartesian} (see Section xxx), except that specifying mesh spacings requires more than just a single \texttt{MFLOAT} value for \texttt{Cartesian( )}. 
You must pass in an array of arrays of \texttt{MFLOAT} values \texttt{(MFLOA T * * )} one array for each dimension, having size given by the number of cells along that dimension. 
The new fourth constructor category adds specification of mesh-spacing boundary conditions. 
You create an array of type \texttt{MeshBC\_E}, an enumeration having values \texttt{\{Reflective,Periodic,None\}}. This tells \texttt{Cartesian} how to provide geometry information such as 
mesh spacings beyond the physical edge of the mesh, as might arise in implementing differential operators using finite differencing of \texttt{Field}'s centered on the mesh. There are 
basically only two ways to do this: wrap around periodically to mesh spacing values inside the physical mesh, or reflect the mesh spacing values across the boundary. The user should 
avoid specifying \texttt{None} for mesh boundary-condition types. The following code example illustrates using this fourth constructor category for \texttt{Cartesian()}, and how to set up 
mesh spacing and boundary condition specifiers: 
\begin{smallcode}

const unsigned Dim = 3U;
unigned nx=5, ny=5, nz=5; 
Index I(nx),J(nz) ,K(nz);

NDIndex<Dim> ndi; 
ndi[0] = I; ndi[1] = J; ndi[2] =K; 

// Defining spacings and origin, use double because it's default of MFLOAT: 
double* spacings[Dim]; 
spacings[0] = new double [nx] ; 
spacings[1] = new double [ny] ; 
spacings[2] = new double [nz];
 
int vert; 
for (vert=0; vert < nx; vert++) (delX[0]) [vert] = 1.0 + vert*1.0;
for (vert=0; vert < ny; vert++) (delX[1]) [vert] = 2.0 + vert*2.0;
for (vert=0; vert < nz; vert++) (delX[2]) [vert] = 3.0 + vert*3.0;

Vektor<double,Dim> origin; 
origin(0) = 5.0; 
origin(1) = 6.0; 
origin(2) = 7.0'

MeshBC_E meshbc[2*Dim];
 
for (unsigned face=0; face < (2*Dim); face++) 
 meshbc[face] =Reflective; 
 
Cartesian<Dim> cmesh(ndi, ,spacings, origin, meshbc); 

\end{smallcode}
The \texttt{cmesh} object has mesh spacings $\Delta_x = {1.,2., \dots}, \Delta_y = {2.,3., \dots} \text{ and } \Delta_z = {3.,6., \dots}$ the origin is at (5.0, 6.0, 7.0) and reflective mesh boundary conditions on all faces. 
The other three cases of \texttt{Cartesian ()} are like this but with fewer parameters. As in \texttt{UniformCartesian}, if the origin is unspecified it defaults to (0,0,0), 
if the mesh spacings are unspecified they default to uniform values of 1.0 for all cells along all axes. The mesh boundary conditions default to \texttt{Reflective} on all faces. 

\subsection{\texttt{Cartesian} Member Functions and Member Data} 
\subsubsection{ \texttt{Cartesian} Member Data for Sizes and Spacings}
The types are all the same as for \texttt{UniformCartesian} (See Appendix \ref{app:unifcart} ). The values of and interpretation of the \texttt{Dvc} data member are completely different; functions such as \texttt{Div()} which 
use this information to implement differential operators must account properly for the variation of mesh spacing values and the difference between cell-cell and vertex-vertex spacings in the nonuniform case. 
For \texttt{UniformCartesian}, the single \texttt{Dvc} array could hold all the necessary information; for \texttt{Cartesian}, a slightly different \texttt{Dvc} coordinates with VertSpacings and CellSpacings, 
which must exist before the user can invoke operators such as \texttt{Div()}.

\subsubsection{\texttt{Cartesian} Set/Accessor Functions for Member Data}
Except for the return value of\texttt{ get\_meshSpacing()} , and the new functions \texttt{get\_MeshBC()} ,the interfaces and descriptionsare all the same as for\\  \texttt{UniformCartesian}. (See Appendix \ref{app:unifcart}).
\begin{smallcode}
MFLOAT* get_meshSpacing(int d);
\end{smallcode}
Returns the array of mesh-spacing values along the specified direction. The number of elements in the arrays the number of cells in the mesh along that dimension. 
\begin{smallcode}
MeshBC_Eget_MeshBC (uns,igned face) ;
\end{smallcode}
Returns the value of the mesh boundary-condition specifier for the requested face of the mesh. The numbering convention for faces is: 0 means low face of 1st dimension, 1 means high face of 1 st dimension, 
2 means low face of 2nd dimension, 3 means high face of 2nd dimension, and so on. 
\begin{smallcode}
MeshBC_E* get_MeshBC(); 
\end{smallcode}
Returns the array of values of mesh boundary-condition specifiers for all faces of the \texttt{Mesh}, following the numbering convention described in the description of the one-argument prototype 
of this function above. The number of elements in the array is two times the number of dimensions. 
 
\subsubsection{Other Cartesian Methods} 
As for \texttt{UniformCartesian}, most of the public member functions in \texttt{Cartesian} are designed to return some typical kinds of geometrical information about  the mesh that a user (programmer or class) 
might want. So far, all of the member functions mirror \texttt{UniformCartesian} exactly, except for the occasional replacement of \\  "\texttt{Cartesian}" for 
"\texttt{UniformCartesian}" in mesh template-parameters for \texttt{Field} arguments.









